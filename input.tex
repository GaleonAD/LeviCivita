\documentclass[a4paper; 11pt]{article}

\bibliographystyle{alpha}


% global includes
\usepackage[polish]{babel}
\usepackage[utf8]{inputenc}
\usepackage{polski}
\usepackage{courier} %times, kurier
\usepackage{amsmath}
%\usepackage{amssymb}
\usepackage{graphicx}
\usepackage{geometry}
\usepackage{indentfirst}
\usepackage{icomma}
\usepackage{booktabs}
\usepackage{float}
\usepackage[locale=FR]{siunitx}
\usepackage{verbatim}
%\sisetup{per-mode=fraction}
%\sisetup{per-mode=reciprocal}
\sisetup{per-mode=symbol}
\newgeometry{tmargin=2.3cm, lmargin=1.9cm, rmargin= 1.9cm, bmargin= 2.3cm}

% local includes
\usepackage{color}
\usepackage{subfig}

\renewcommand{\figurename}{Rys.}
\renewcommand{\tablename}{Tab.}
\renewcommand{\abstractname}{Abstrakt}
\renewcommand{\d}{\text{d}}
\newcommand{\D}{\text{D}}

\title{On Levi-Civita pseudotensor}
\author{Paweł Rzońca}
\date{\today}
%\date{}

\begin{document}

\maketitle
%%%%%%%%%%%%%%%%%%%%%%%%%%%%%%%%%%%%%%%%%%%%%%%%%%%%%%%%%%%%%%%%%%%%%%%%%%%%%%%
% LeviCivita
%%%%%%%%%%%%%%%%%%%%%%%%%%%%%%%%%%%%%%%%%%%%%%%%%%%%%%%%%%%%%%%%%%%%%%%%%%%%%%%
In this paper we present properties of Levi-Civita pseudotensor.
Let us firstly denote $n$ as dimention of space.
We define covariant Levi-Civita pseudorensor as follows
\begin{equation}
\varepsilon^{i_1\dots i_n} = 
\left\{ 
\begin{array}{cl}
1 & \text{if}\ (i_1,\ \dots, i_n)\ \text{is an even permutation}\\
-1 & \text{if}\ (i_1,\ \dots, i_n)\ \text{is an odd permutation}\\
0 & \text{if}\ i_p = i_q \ \text{for some}\ p,\ q \\
\end{array}
\right.
\end{equation}
and consistently contravaraint
\begin{equation}\label{contravariant}
\varepsilon_{i_1\dots i_n} = 
g_{i_1 j_1} \dots g_{i_n j_n} \varepsilon^{j_1\dots j_n} ,
\end{equation}
where $g_{ij}$ is metric tensor $(g^{ik} g_{kj} = \delta^i_j)$ and 
$\sqrt{-g}=1$. One can easly see that we can, 
using equality \eqref{contravariant}, define determinant of 
matrix $M$
\begin{equation}
\det ( M ) \varepsilon_{i_1\dots i_n} = 
M_{i_1 j_1} \dots M_{i_n j_n} \varepsilon^{j_1\dots j_n} .
\end{equation}
It is worth mentioning that in curved space unit antisimetric pseudotensor 
$E^{i_1\dots i_n}$ will be definied as follows
\begin{equation}
E^{i_1\dots i_n} = \frac{1}{\sqrt{-g}} e^{i_1\dots i_n}
\end{equation}
\begin{equation}
E_{i_1\dots i_n} =\sqrt{-g} e_{i_1\dots i_n}
\end{equation}
Now we will present usefull properties of Levi-Civita psedotensor.
\begin{align}
\varepsilon^{i_1\dots i_n}\varepsilon_{i_1\dots i_n} = n!
\end{align}
\begin{align}
\varepsilon^{i_1 i_2\dots i_n}\varepsilon_{j_1 i_2 \dots i_n} =
(n-1)!\ \delta^{i_1}_{j_1} 
\end{align}


%%%%%%%%%%%%%%%%%%%%%%%%%%%%%%%%%%%%%%%%%%%%%%%%%%%%%%%%%%%%%%%%%%%%%%%%%%%%%%%
% 2D
%%%%%%%%%%%%%%%%%%%%%%%%%%%%%%%%%%%%%%%%%%%%%%%%%%%%%%%%%%%%%%%%%%%%%%%%%%%%%%%
\newpage
\textbf{2 dimensions}
\\
\begin{align}
(\varepsilon^{ij})=
\left(
\begin{array}{cc}
 0 & 1 \\
  -1 & 0 \\
\end{array}
\right)
\end{align}
\begin{align}
(\varepsilon_{ij})=
\left(
\begin{array}{cc}
 0 & -1 \\
  1 & 0 \\
\end{array}
\right)
\end{align}
\begin{align}
\varepsilon^{ij}\varepsilon_{pq} = \delta^i_p \delta^j_q - \delta^i_q \delta^j_p
\end{align}
\begin{align}
\varepsilon^{ij}\varepsilon_{jk} = \delta^i_k
\end{align}
\begin{align}
\varepsilon^{ij}\varepsilon_{ij} = 2
\end{align}



%%%%%%%%%%%%%%%%%%%%%%%%%%%%%%%%%%%%%%%%%%%%%%%%%%%%%%%%%%%%%%%%%%%%%%%%%%%%%%%
% 3D
%%%%%%%%%%%%%%%%%%%%%%%%%%%%%%%%%%%%%%%%%%%%%%%%%%%%%%%%%%%%%%%%%%%%%%%%%%%%%%%
\newpage
\textbf{3 dimensions}
\\
\begin{align}
\varepsilon^{ijk}\varepsilon_{pqr} = 
\delta^i_p \delta^j_q \delta^k_r
+\delta^i_q \delta^j_r \delta^k_p
+\delta^i_r \delta^j_p \delta^k_q
-\delta^i_p \delta^j_r \delta^k_q
-\delta^i_r \delta^j_q \delta^k_p
-\delta^i_q \delta^j_p \delta^k_r
\end{align}
\begin{align}
\varepsilon^{ijk}\varepsilon_{ipq} = \delta^j_p \delta^k_q - \delta^j_q \delta^k_p
\end{align}
\begin{align}
\varepsilon^{ipq}\varepsilon_{jpq} =2 \delta^i_j
\end{align}
\begin{align}
\varepsilon^{ijk}\varepsilon_{ijk} = 6
\end{align}

\begin{comment}
%%%%%%%%%%%%%%%%%%%%%%%%%%%%%%%%%%%%%%%%%%%%%%%%%%%%%%%%%%%%%%%%%%%%%%%%%%%%%%%
% 4D
%%%%%%%%%%%%%%%%%%%%%%%%%%%%%%%%%%%%%%%%%%%%%%%%%%%%%%%%%%%%%%%%%%%%%%%%%%%%%%%
\newpage
\textbf{4 dimensions}
\\
\begin{align}
\varepsilon^{ipqr}\varepsilon_{jpqr} =6 \delta^i_j
\end{align}
\begin{align}
\varepsilon^{ijkl}\varepsilon_{ijkl} = 24
\end{align}

\end{comment}


\end{document}


